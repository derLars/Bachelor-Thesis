\newcommand*{\Introduction}{\begingroup

%----------------------------------------------------------------------------------------
%	Introduction
%----------------------------------------------------------------------------------------
\chapter{Introduction}\label{ch:introduction}
\noindent Desing patterns provide solutions for many challenges within software development. Embedded reactive systems show many of these challenges regarding execution time, memory consumption or safety critical behavior among others. Many sources offer well designed ways of implementations in C++ for commonly used design patterns. But most of these sources avoid the use of the new C++11 and C++14 standards with no named reason. The ambition of this document is the implementation of the factory method pattern and the state pattern as a representative set of design patterns which are commonly used within the development of embedded software with the use of the new C++ facilities to identify advantages and disadvantages of the use of the new C++ standards and find out whether it is beneficial or not. 

%----------------------------------------------------------------------------------------
%	Structure
%----------------------------------------------------------------------------------------
\section{Structure}\label{sec:structure}
\noindent The document's structure is devided into three main parts. The first part, including chapter \ref{ch:introduction} to \ref{ch:designPatterns}, provides a general overview and introduction of the main topics for creating a same basis of knowledge regarding embedded reactive systems, embedded software development as well as design patterns. 

\noindent\\ Following this, the main body is divided into the factory method pattern and the state pattern. Each of these chapters explains  several ways of implementation for both C++03 and C++11 including the particular advantages and disatvantages, an introduction of the used facilities of the new C++ standards and an analysis of the particular ways of implementation.

\noindent\\ Chapter \ref{ch:conclusion} and \ref{ch:perspective} finalize the work by summarizing the results of the analysis and providing an opinion and a perspective based on the summary.

%For providing a basic overview of the main topics and the environment the document starts with a short introduction of embedded reactive systems (Chapter \ref{ch:embeddedReactiveSystems}), the use of C++ within embedded software (Chapter \ref{ch:cppForEmbeddedSoftware}) and the introduction of design patterns (Chapter \ref{ch:designPatterns}). Following this, the main body is divided into the factory method pattern (Chapter \ref{}) where the conditional statement implementation
\endgroup}
